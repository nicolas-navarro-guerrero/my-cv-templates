%******************************************************************************
%\title MY-CV-TEMPLATE_A4_EN.tex
%\brief
% 
% Original by Xavier Danaux (xdanaux@gmail.com)
%             http://www.ctan.org/pkg/moderncv
% Derivated by 2013 Nicolas Navarro-Guerrero
% Contact:	nicolas.navarro.guerrero@gmail.com
%		https://github.com/nicolas-navarro-guerrero/
% Created in: October 2013
% Last Modification by: Nicolas Navarro-Guerrero
% Last Modification in: October 2013
%
% LICENSE:
%  This file is part of
%  "my-cv-templates"
%
%  "my-cv-templates"
%  is free software: you can redistribute it and/or modify it under the terms of
%  the GNU General Public License as published by the Free Software Foundation,
%  either version 3 of the License, or (at your option) any later version.
%
%  "my-cv-templates"
%  is distributed in the hope that it will be useful, but WITHOUT ANY WARRANTY;
%  without even the implied warranty of MERCHANTABILITY or FITNESS FOR A
%  PARTICULAR PURPOSE. See the GNU General Public License for more details.
%
%  You should have received a copy of the GNU General Public License along with
%  "my-cv-templates".
%  If not, see <http://www.gnu.org/licenses/>.
%******************************************************************************

% possible options include font size ('10pt', '11pt' and '12pt'),
% paper size ('a4paper', 'letterpaper', 'a5paper', 'legalpaper',
% 'executivepaper' and 'landscape') and font family ('sans' and 'roman')
\documentclass[11pt,a4paper,sans]{mymoderncv}

\usepackage[spanish,english,french,german]{babel}
% Date will be language-dependent
\usepackage[nodayofweek]{datetime}

% my moderncv themes
% style options are 'casual' (default), 'classic', 'oldstyle', 'banking',
% 'bankingalt' and 'custom'
\mymoderncvstyle{classic}
% color options 'blue' (default), 'orange', 'green', 'red', 'purple', 'grey',
% and 'pinegreen'
\mymoderncvcolor{blue}
% to set the default font; use '\sfdefault' for the default sans serif font,
% '\rmdefault' for the default roman one, or any tex font name
%\renewcommand{\familydefault}{\sfdefault}
% uncomment to suppress automatic page numbering for CVs longer than one page
%\nopagenumbers{}

% character encoding
% if you are not using xelatex ou lualatex, replace by the encoding you are using
\usepackage[utf8]{inputenc}

% adjust the page margins
%\usepackage[scale=0.75]{geometry}
\usepackage[top=2cm, bottom=2cm, left=2.5cm, right=2.5cm]{geometry}
% if you want to change the width of the column with the dates
%\setlength{\hintscolumnwidth}{3cm}
% for the 'classic' style, if you want to force the width allocated to your
% name and avoid line breaks. be careful though, the length is normally
% calculated to avoid any overlap with your personal info; use this at your own
% typographical risks...
%\setlength{\makecvtitlenamewidth}{10cm}

%%+++++++++++++++++++++++++++++++++++++++++++++++++++++++++++++++++++++++++++++
% PERSONAL DATA
\firstname{John}
\familyname{Doe}
%\name{Nicol\'as}{Navarro-Guerrero}
% optional, remove / comment the line if not wanted
\title{Resum\'e title}
% optional, remove / comment the line if not wanted; the "postcode city" and
% "country" arguments can be omitted or provided empty
\address{street and number}{postcode city}{country}
% optional, remove / comment the line if not wanted; the optional "type" of the
% phone can be "mobile" (default), "fixed" or "fax"
\phone[mobile]{+1~(234)~567~890}
% OPTIONAL FIELDS, remove / comment the line if not wanted
\email{john@doe.org}
\homepage{www.johndoe.com}
\extrainfo{additional information (nationality)}
% '64pt' is the height the picture must be resized to, 0.4pt is the thickness
% of the frame around it (put it to 0pt for no frame) and 'picture' is the
% name of the picture file
%\photo[100pt][0.4pt]{placeholder.png}
\quote{Some quote}

% to show numerical labels in the bibliography (default is to show no labels);
% only useful if you make citations in your resume
%\makeatletter
%\renewcommand*{\bibliographyitemlabel}{\@biblabel{\arabic{enumiv}}}
%\makeatother
% CONSIDER REPLACING THE ABOVE BY THIS
%\renewcommand*{\bibliographyitemlabel}{[\arabic{enumiv}]}

% bibliography with mutiple entries
%\usepackage{multibib}
%\newcites{book,misc}{{Books},{Others}}
%------------------------------------------------------------------------------
%            content
%------------------------------------------------------------------------------
\begin{document}
%------------------------------------------------------------------------------
%-----       resume       -----------------------------------------------------
%------------------------------------------------------------------------------
\makecvtitle

%%+++++++++++++++++++++++++++++++++++++++++++++++++++++++++++++++++++++++++++++
\section{Education}
% arguments 3 to 6 can be left empty
\cventry{year--year}{Degree}{Institution}{City}{\textit{Grade}}{Description}
\cventry{year--year}{Degree}{Institution}{City}{\textit{Grade}}{Description: This item is particularly long and therefore normally spans over several lines. Did you notice the indentation when the line wraps?}

%%+++++++++++++++++++++++++++++++++++++++++++++++++++++++++++++++++++++++++++++
\section{Master thesis}
\cvitem{title}{\emph{Title}}
\cvitem{supervisors}{Supervisors}
\cvitem{description}{This item is particularly long and therefore normally spans over several lines. Did you notice the indentation when the line wraps?}

%%+++++++++++++++++++++++++++++++++++++++++++++++++++++++++++++++++++++++++++++
\section{Experience}
\subsection{Vocational}
\cventry{year--year}{Job title}{Employer}{City}{}{General description no longer than 1--2 lines.\newline{}
Detailed achievements:
\begin{itemize}
\item Achievement 1;
\item Achievement 2, with sub-achievements:
  \begin{itemize}
  \item Sub-achievement (a);
  \item Sub-achievement (b), with sub-sub-achievements (don't do this!);
    \begin{itemize}
    \item Sub-sub-achievement i;
    \item Sub-sub-achievement ii;
    \item Sub-sub-achievement iii;
    \end{itemize}
  \item Sub-achievement (c);
  \end{itemize}
\item Achievement 3.
\end{itemize}}
\cventry{year--year}{Job title}{Employer}{City}{}{Description line 1\newline{}Description line 2}
\subsection{Miscellaneous}
\cventry{year--year}{Job title}{Employer}{City}{}{This item is particularly long and therefore normally spans over several lines. Did you notice the indentation when the line wraps?}

%%+++++++++++++++++++++++++++++++++++++++++++++++++++++++++++++++++++++++++++++
\section{Languages}
\cvitemwithcomment{Language 1}{Skill level}{This item is particularly long and therefore normally spans over several lines. Did you notice the indentation when the line wraps?}
\cvitemwithcomment{Language 2}{This item is particularly long and therefore normally spans over several lines. Did you notice the indentation when the line wraps?}{Comment}
\cvitemwithcomment{Language 3}{Skill level}{Comment}

%%+++++++++++++++++++++++++++++++++++++++++++++++++++++++++++++++++++++++++++++
\section{Computer skills}
\cvdoubleitem{category 1}{XXX, YYY, ZZZ}{category 4}{XXX, YYY, ZZZ}
\cvdoubleitem{category 2}{XXX, YYY, ZZZ}{category 5}{XXX, YYY, ZZZ}
\cvdoubleitem{category 3}{This item is particularly long and therefore normally spans over several lines. Did you notice the indentation when the line wraps?}{}{}

%%+++++++++++++++++++++++++++++++++++++++++++++++++++++++++++++++++++++++++++++
\section{Interests}
\cvitem{hobby 1}{Description}
\cvitem{hobby 2}{This item is particularly long and therefore normally spans over several lines. Did you notice the indentation when the line wraps?}
\cvitem{hobby 3}{Description}

%%+++++++++++++++++++++++++++++++++++++++++++++++++++++++++++++++++++++++++++++
\section{Extra 1}
\cvlistitem{Item 1}
\cvlistitem{Item 2}
\cvlistitem{Item 3. This item is particularly long and therefore normally spans over several lines. Did you notice the indentation when the line wraps?}

%%+++++++++++++++++++++++++++++++++++++++++++++++++++++++++++++++++++++++++++++
\section{Extra 2}
\cvlistdoubleitem{Item 1}{Like item 3 in the single column list before, this item is particularly long to wrap over several lines.}
\cvlistdoubleitem{Item 2}{Item 5\cite{book1}}
\cvlistdoubleitem{Item 3. Like item 3 in the single column list before, this item is particularly long to wrap over several lines.}{}

%%+++++++++++++++++++++++++++++++++++++++++++++++++++++++++++++++++++++++++++++
\section{References}
\begin{cvcolumns}
  \cvcolumn{Category 1}{\begin{itemize}\item Person 1\item Person 2\item Person 3\end{itemize}}
  \cvcolumn{Category 2}{Amongst others:\begin{itemize}\item Person 1, and\item Person 2\end{itemize}(more upon request)}
  \cvcolumn[0.5]{All the rest \& some more}{\textit{That} person, and \textbf{those} also (all available upon request).}
\end{cvcolumns}

% Publications from a BibTeX file without multibib
% CONSIDER MERGING WITH PREAMBLE PART
%  for numerical labels:
%      \renewcommand{\bibliographyitemlabel}{\@biblabel{\arabic{enumiv}}}
%  to redefine the heading string ("Publications"):
%      \renewcommand{\refname}{Articles}
\nocite{*}
\bibliographystyle{plain}
% 'publications' is the name of a BibTeX file
\bibliography{publications}

% Publications from a BibTeX file using the multibib package
%\section{Publications}
%\nocitebook{book1,book2}
%\bibliographystylebook{plain}
% 'publications' is the name of a BibTeX file
%\bibliographybook{publications}
%\nocitemisc{misc1,misc2,misc3}
%\bibliographystylemisc{plain}
% 'publications' is the name of a BibTeX file
%\bibliographymisc{publications}

\begin{flushright}
\vspace{5em}
\selectlanguage{english}
%\selectlanguage{german}
%\selectlanguage{french}
%\selectlanguage{spanish}
City, Country - \today
\end{flushright}
%\clearpage
\end{document}